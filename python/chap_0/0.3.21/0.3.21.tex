\documentclass[11pt,a4paper]{jsarticle}
%
\usepackage{amsmath,amssymb}
\usepackage{bm}
\usepackage{graphicx}
\usepackage{ascmac}
%
\setlength{\textwidth}{\fullwidth}
\setlength{\textheight}{39\baselineskip}
\addtolength{\textheight}{\topskip}
\setlength{\voffset}{-0.5in}
\setlength{\headsep}{0.3in}
%
\newcommand{\divergence}{\mathrm{div}\,}  %ダイバージェンス
\newcommand{\grad}{\mathrm{grad}\,}  %グラディエント
\newcommand{\rot}{\mathrm{rot}\,}  %ローテーション
%

\title{\bf 「行列プログラマー」p.10 \\ 0.3.21解答}
\author{RyodoTanaka}
\date{}
\begin{document}
%
%
\maketitle

\section*{問題}
補題0.3.21を証明せよ.

\section*{解答}

\begin{eqnarray}
g:U→V,f:V→W 
\end{eqnarray}

とする.\\
このとき,命題0.3.12より

\begin{eqnarray}
 (g^{-1}\circ f^{-1})\circ (f\circ g)&=&g^{-1}\circ (f^{-1}\circ (f\circ g)) \nonumber \\
 &=&g^{-1}\circ ((f^{-1}\circ f)\circ g) \nonumber \\
 &=&g^{-1}\circ ({\rm id}_V\circ g) \nonumber \\
 (g^{-1}\circ f^{-1})\circ (f\circ g)&=&g^{-1}\circ g={\rm id}_U\label{032450_7Jun17} \\
 \nonumber \\
 (f\circ g)\circ (g^{-1}\circ f^{-1})&=&f\circ (g\circ (g^{-1}\circ f^{-1})) \nonumber\\
 &=&f\circ ((g\circ g^{-1})\circ f^{-1}) \nonumber \\
 &=&f\circ ({\rm id}_V\circ f^{-1}) \nonumber \\
 (f\circ g)\circ (g^{-1}\circ f^{-1})&=&f\circ f^{-1}={\rm id}_W\label{032550_7Jun17}
\end{eqnarray}

となる.\\
よって,(\ref{032450_7Jun17}),(\ref{032550_7Jun17})より,$g^{-1}\circ f^{-1}=(f\circ g)^{-1}$が言える.

\end{document}
